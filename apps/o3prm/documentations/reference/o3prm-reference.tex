% -*- coding: UTF-8 -*-
\documentclass[12pt,a4paper]{article}

\usepackage{reference}

\title{O3PRM Language Specification}
\author{Lionel Torti \\ Pierre-Henri Wuillemin \\ Laboratoire d'Informatique de Paris VI - UPMC}

\begin{document}

\maketitle

The O3PRM language purpose is to model \PRMs{} using a strong \oo{} syntax. This
document purpose it not to explain \PRMs{} but to provide a complete
specification of the O3PRM language. Tutorials and examples of \PRMs{} can be
found on the \href{http://o3prm.lip6.fr/}{O3PRM website}.

\section{O3PRM project structure}

As in Java, the O3PRM language is made of \e{compilation units} that are placed
in \e{packages}. It is possible to encode in a single file an entire project but
it is not recommended. A package matches a specific directory and in each file
we can find at least one compilation unit. The following is a sample O3PRM
project:

\begin{small}
\begin{verbatim}
fr\
  | lip6\
  |     | printers\
  |     | types.o3prm       // types definition
  |     | powersupply.o3prm // class PowerSupply definition
  |     | equipment.o3prm   // class Equipment definition
  |     | room.o3prm        // class Room definition
  |     | computer.o3prm    // class Computer definition
  |     | printer.o3prm     // class Printer definition
  |     | example.o3prm     // system Example definition
  |     | query.o3prmr      // request query definition
\end{verbatim}
\end{small}

File extensions can be used as indicator of the file's compilation unit nature.
The following extensions are allowed: \verb+.o3prmr+ for queries,  \verb+o3prm+
for everything else (types, classes, interfaces and systems).  It is good
pratice to name a file with the compilation unit it holds, for example in
\verb+computer.o3prm+ should contain the definition for the \verb+Computer+
class. If the file contains several compilation units, you should use the most
hight order unit to name the file. For example, if you define types, classes
and a system in one file, you should use the system's name.

\subsection*{Compilation units}

There exists four different sorts of compilation units. A compilation unit
\e{declares} a specific element in the modeling process and can either be: an
attribute's type, a class, an interface, a system or a query. Each compilation
unit can start with a header. Headers are where you declare imports statements.

\begin{footnotesize}
\begin{verbatim}
<O3PRM> ::= [<header>] <compilation_unit> [(<compilation_unit>)]
<header> ::= <import>
<compilation_unit> ::= <type_unit> |
                       <class_unit> |
                       <system_unit> |
                       <query_unit>
\end{verbatim}
\end{footnotesize}

\subsection*{Header syntax}

Each compilation unit is declared in a namespace defined by the path from the
project's root to the file's name in wich it is declared. Directory separators
are represented using dots. For example the file
\verb+fr/lip6/pritners/types.o3prm+ defines the namespace
\verb+fr.lip6.printers.types+.

Namespaces can be used to import all of the compilation units defined in them,
since many compilation units will need units defined in other files.  In such
cases, we say that a given compilation unit has dependencies which are declered
using the \verb+import+ keyword. The syntax is:

\begin{footnotesize}
\begin{verbatim}
<import> ::= import <path> ";"
<path> ::= <word> [ ( "." <word> ) ]
<word> ::= <letter> (<letter> | <digit>)
<letter> ::= 'A'..'Z' + 'a'..'z'+ '_'
<integer> ::= <digit> <digit>*
<float> ::= <integer> "." <integer>
<digit>  ::= '0'..'9'
\end{verbatim}
\end{footnotesize}

An example:

\begin{footnotesize}
\begin{verbatim}
import fr.skoob.printers.computer;
\end{verbatim}
\end{footnotesize}

The \verb+import+ instruction is made of a module's name.

The O3PRM interpreter should use an environment variable to know wich directory
to look up for resolving compilations units. It is recommanded that it behaves
like the \verb+CLASSPATH+ variable used by Java compilators.

Compilation units can be accessed through two names:
\begin{itemize}
\item Its simple name, as it is declared (for example \verb+Computer+).
\item Its full name, defined by its namespace and its declaration (for exampe
\verb+fr.lip6.printers.Computer+).
\end{itemize}
In most case, refering to a compilation unit using its simple name will work,
you will need full names only to prevent name collision. Name collisions happen
when two compilation units have the same name but are declared in different
namespaces. In such sitatuation the O3PRM interpreter cannot resolve the name
and must raise an interpretation error.

Note that no matter how you refer to a compilation unit (either by its simple
name or full name) you must always import it's namespace.

\section{Attribute type declaration}

The O3PRM language can only use discrete \rvs{} that are either user-defined or
one of the three built-in types: \verb+boolean+, \verb+int+ and \verb+real+.
User-defined types are declared using the keyword \verb+type+:

\begin{footnotesize}
\begin{verbatim}
<type_unit> ::= <built_in> | <user_defined>
<user_defined> ::= <basic_type> | <subtype>
<basic_type> ::= type <word> <word> "," ( "," <word> )+ ";"
\end{verbatim}
\end{footnotesize}

The rule \verb+<basic_type>+ defines a \rv{}'s domain, the first \verb+word+
is the type's name and the following are the domain label's names. There must
be at least two labels. The rule \verb+<subtype>+ is explained in the
next section. Some examples:

\begin{footnotesize}
\begin{verbatim}
boolean exists;
int (0,9) power;
real (0, 90, 180) angle;
type t_state OK, NOK;
\end{verbatim}
\end{footnotesize}

\subsection*{Subtyping}

A subtype can be declared using the \verb+extends+ keyword. A subtype
declaration syntax is:

\begin{footnotesize}
\begin{verbatim}
<subtype> ::= type <word> extends <word> <type_spec>
<type_spec> ::= <word> ":" <word> (<word> ":" <word>)+
\end{verbatim}
\end{footnotesize}

The first \verb+<word>+ is the type's name, the second the name of its
supertype and the rule \verb+<type_spec>+ defines label specializations:
the first \verb+<word>+ is the subtype's label and the second \verb+<word>+
is the supertype's label. A example of subtype declaration:

\begin{footnotesize}
\begin{verbatim}
type t_degraded extends t_state
OK: OK,
DYSFONCTION: NOK,
DEGRADED: NOK;
\end{verbatim}
\end{footnotesize}

In this example, \verb+DYSFONCTION+ and \verb+DEGRADED+ are specializations of
the label \verb+NOK+ of type \verb+t_state+.  When declaring a subtype, it
is mandatory that the supertype is visible, \ie{}:

\begin{itemize}
\item either the supertype was declared in the same file before the subtype;
\item or the supertype declaration unit has been imported.
\end{itemize}

\subsection*{Built-in types}

The built-in types are: \verb+boolean+, \verb+int+ and \verb+real+.  The
\verb+boolean+ type is used to represent binary \rvs{} taking the values
\verb+false+ or \verb+true+. Note that the order used is always \verb+false+
first and then \verb+true+.  The \verb+int+ must be used to define \rvs{}
over ranges: \verb+int(0,9) power+ defines a \rv{} power over the domain
integers from $0$ to $9$.  The \verb+real+ type must be used to define \rvs{}
discretized over continuous domains. For example, \verb+real(0, 90, 180) angle+
defines a \rv{} with the two values \verb+[0-90[+, \verb+[90,180[+.
When defining a \rv{} with the \verb+real+ type there must be at least three
parameters. The syntax of built-in types is:

\begin{footnotesize}
\begin{verbatim}
<built_in> ::=
  boolean <word> ";" |
  int "(" <digit>* "," <digit>* ")" <word> ";" |
  real "(" <digit>* ( "," <digit>* )+ ")" <word> ";"
\end{verbatim}
\end{footnotesize}

\newpage
\section{Class and interface declaration}

Classes and interfaces are declared as follow:

\begin{footnotesize}
\begin{verbatim}
<class_unit> ::= <class> | <interface>
<class> ::= class <word> [ extends <word> ] "{" <class_elt>* "}"
<class_elt> ::= <reference_slot> | <attribute> | <parameter>
<interface> ::= interface <word> [ extends <word> ]
                "{" <interface_elt> "}"
<interface_elt> ::= <reference_slot> | <abstract_attr>
\end{verbatim}
\end{footnotesize}

The first \verb+word+ is the class name and the second (if any) the class
superclass. An example:
\begin{footnotesize}
\begin{verbatim}
class A {
  // reference slots and attributes declaration
}

class B extends A {
  // This class extends class A
  // We say that B is a subclass of A
  // And A the superclass of B.
}
\end{verbatim}
\end{footnotesize}

\subsection*{Reference slot declaration}

In the O3PRM language, simple (1 to 1) and complex (1 to N) reference slots are
declared differently. Simple reference slots can only refer to a single instance
and complex reference are arrays. The syntax for declaring a reference slot is:

\begin{footnotesize}
\begin{verbatim}
<reference_slot> ::= <word> [ "[" "]" ] <word> ";"
\end{verbatim}
\end{footnotesize}

The first \verb+word+ is the reference slot range's name, we do not called it
it's \emph{type} since it could be ambiguous with attributes types. If it is
complex \verb+[ ]+ are added as suffixes to the range's name and the last
\verb+word+ is the reference slot's name. The following is an example of two
reference slots declaration:

\begin{footnotesize}
\begin{verbatim}
// Simple reference slot
A refA;
// Complex reference slot
B[] refB;
\end{verbatim}
\end{footnotesize}

\subsection*{Attribute declaration}

Attributes are declared as follows:

\begin{footnotesize}
\begin{verbatim}
<attribute> ::= <word> <word> [ dependson <parents> ]
                ( <CPT> | <function> ) ";"
<parents> ::= <path> ( "," <path> )*
<CPT> ::= "{" ( <raw_CPT> | <rule_CPT> ) "}"
\end{verbatim}
\end{footnotesize}

The first \verb+word+ is the attribute's type, the second its name.
Dependencies are defined as a list of parents separated by commas. Each parent
is defined by a \verb+<path>+, \ie{} a list of reference slots ending by an
attribute. We will detail \cpts{} declaration in the next section. Functions
will be detailed in section \ref{O3PRMFunc}. The following is an example of
attribute declaration:

\begin{footnotesize}
\begin{verbatim}
// An attribute with no parents
a_type a_name {
  cpt_declaration
};
// An attribute with two parents
another_type another_name dependson parent_1, parent_2 {
  cpt_declaration
};
\end{verbatim}
\end{footnotesize}

\subsection*{Raw \cpt{} declaration}

When declaring a raw \cpt{}, all values of the \cpt{} must be given. In such
cases, the value's order is paramount. The declaration used in O3PRM is by
columns, \ie{} each column in the \cpt{} must sum to one. Let us consider the
boolean attributes $X$, $Y$ and $Z$ such that $X$ depends on $Y$ and $Z$. The
first value in $X$'s \cpt{} declaration will be the probability
$P(X=false|Y=false,Z=false)$ and the next value is done by increasing the
domain of the last attribute by one. In this case, the second value is the
probability $P(X=false|Y=false,Z=true)$. When the last attribute reached its
last value, we set it back to its first value and increase the previous
attribute. For example, the third value of $X$'s \cpt{} would be the
probability $P(X=false|Y=true,Z=false)$. The following illustrates how we can
use comments to make raw \cpt{} definitions easier to read.

\begin{footnotesize}
\begin{verbatim}
boolean X dependson Y, Z {
  // Y=       |     false    |     true     |
  // Z=       | false | true | false | true |
  /* false */ [ 1.0,    0.3,   1.0,    0.01,
  /* true  */   0.0,    0.7,   0.0,    0.99 ]
};
\end{verbatim}
\end{footnotesize}

The \cpt{} declaration is dependent on the order in which the parents are
declared. The syntax of a raw \cpt{} declaration is straightforward:

\begin{footnotesize}
\begin{verbatim}
<raw_CPT> ::= "[" <float>+ ( "," <float>+ )+ "]"
\end{verbatim}
\end{footnotesize}

\subsection*{Rule based \cpt{} declaration}

Rule based declarations exploit wildcards to reduce the number of parameters
for \cpt{} with redundant values. Its syntax is:

\begin{footnotesize}
\begin{verbatim}
<rule_CPT> ::= ( <word> ("," <word>)* ":" <float> ";" )+
\end{verbatim}
\end{footnotesize}

There is no limit in the number of rules and when two rules overlap the last
rule takes precedence. The following is an example of rule based declaration
using the previous example:

\begin{footnotesize}
\begin{verbatim}
boolean X dependson Y, Z {
// Y,     Z:     X=false, X=true
   *,     false: 1.0,     0.0;
   true,  true:  0.01,    0.99;
   false, true:  0.3,     0.7;
};
\end{verbatim}
\end{footnotesize}

\subsection*{Parameters}

Parameters are declared using the following syntax:

\begin{footnotesize}
\begin{verbatim}
<parameter> ::= <int_parameter> | <real_parameter>
<int_parameter> ::= "int" <word> default <integer> ";"
<real_parameter> ::= "real" <word> default <float> ";"
\end{verbatim}
\end{footnotesize}

Parameters are either real or integers constants. They are used in CPT
definitions to define values using formulas. Some examples:

\begin{footnotesize}
\begin{verbatim}
// An integer parameter
int t = 3600;
// A real parameter
real lambda default 0.003;
\end{verbatim}
\end{footnotesize}

A parameter value can be set at instantiation.

\subsection*{Formulas}

It is possible to use formulas in place of float numers to define an attribute's CPT.
The formula must be enclosed between quotes and its syntax is implementation specific.
For example:

\begin{footnotesize}
\begin{verbatim}
class A {
  int t = 3600;
  real lambda = 0.003;

  bool state { [ "1 - exp(-lambda*t)", "exp(-lambda*t)" ] };
}
\end{verbatim}
\end{footnotesize}

\subsection*{Interface's abstract attributes}

An abstract attribute in an interface declaration syntax is:

\begin{footnotesize}
\begin{verbatim}
<abstract_attr> ::= <word> <word> ";"
\end{verbatim}
\end{footnotesize}

Where the first \verb+<word>+ is the abstract attribute's type and the second
its name.

\section{System declaration}

A system is declared as follow:

\begin{footnotesize}
\begin{verbatim}
<system> ::= system <word> "{" <system_elt>* "}"
<system_elt> ::= <instance> | <affectation>
\end{verbatim}
\end{footnotesize}

The first \verb+word+ is the system's name. A system is composed of instance
declarations and affectations. Affectations assign an instance to an instance's
reference slot.  The following illustrates a system declaration:

\begin{footnotesize}
\begin{verbatim}
system name {
  // body
}
\end{verbatim}
\end{footnotesize}

\subsection*{Instance declaration}

The syntax to declare an instance in a system is:

\begin{footnotesize}
\begin{verbatim}
<instance> ::= <word> [ "[" digit* "]" ] <word> ";"
\end{verbatim}
\end{footnotesize}

The first \verb+word+ is the instance's class and the second its name.  For
example, if we have a class \verb%A% we could declare the following instance:

\begin{footnotesize}
\begin{verbatim}
A an_instance;
\end{verbatim}
\end{footnotesize}

We may want to declare arrays of instances. To do so we need to add \verb+[n]+
as a suffix to the instance's type, where \verb+n+ is the number of instances
already added in the array. if \verb+n = 0+ then we can simply write \verb+[]+.

\begin{footnotesize}
\begin{verbatim}
// An empty array of instances
A_class[] a_name;
// A array of 5 instances
A_class[5] another_name;
\end{verbatim}
\end{footnotesize}

You can also specify values for parameters when instantiating a class. The syntax to do so is:

\begin{footnotesize}
\begin{verbatim}
<instance> ::= <word> <word> "(" <parameters> ")" ";"
<parameters> ::= parameter ("," instanceParameter)*
<instanceParameter> ::= <word>"="(<integer>|<float>)
\end{verbatim}
\end{footnotesize}

An example:

\begin{footnotesize}
\begin{verbatim}
// We declare an instance of A_class where a_param equals 0.001
A_class a_name(a_param=0.001);
\end{verbatim}
\end{footnotesize}

\subsection*{Affectation}

\begin{footnotesize}
\begin{verbatim}
<affectation> ::= <path> += <word> ";" |
                  <path> = <word> ";"
\end{verbatim}
\end{footnotesize}

It is possible to add instances into an array, using the \verb-+=- operator:

\begin{footnotesize}
\begin{verbatim}
// Declaring some instances
A_class x;
A_class y;
A_class z;
// An empty array of instances
A_class[] array;
// Adding instances to array
array += x;
array += y;
array += z;
\end{verbatim}
\end{footnotesize}

Reference affectation is done using the \verb+=+ operator:

\begin{footnotesize}
\begin{verbatim}
class A {
  boolean X {[0.5, 0.5]};
}

class B {
  A myRef;
}

system S {
  // declaring two instances
  A a;
  B b;
  // Affecting b's reference to a
  b.myRef = a;
}
\end{verbatim}
\end{footnotesize}

In the case of multiple references, we can either use the \verb+=+ to affect
an array or the \verb-+=- operator to add instance one by one:

\begin{footnotesize}
\begin{verbatim}
class A {
  boolean X {[0.5, 0.5]};
}

class B {
  A myRef[];
}

system S1 {
  // declaring an array of five instances of A.
  A[5] a;
  // declaring an instance of B
  B b;
  // Affecting b's reference to a
  b.myRef = a;
}
// An alternative declaration
system S2 {
  // declaring three instances of A
  A a1;
  A a2;
  A a3;
  // declaring an instance of B
  B b;
  // Affecting b's reference to a
  b.myRef += a1;
  b.myRef += a2;
  b.myRef += a3;
}
\end{verbatim}
\end{footnotesize}

\section{Query unit declaration}

A query unit is defined using the keyword \verb+request+. Its syntax is the
following:

\begin{footnotesize}
\begin{verbatim}
<query_unit> ::= request <word> "{" <query_elt>* "}"
<query_elt> ::= <observation> | <query>
<observation> ::= ( <path> = <word> ) |
                  ( unobserved <path> )
                  ";"
<query> ::= "?" <path> ";"
\end{verbatim}
\end{footnotesize}

The first \verb+word+ is the query's name.  In a query unit we can alternate
between observations and queries. An observation observe an attribute with a
given value. Evidence are affected using the \verb+=+ operator. A query over
attribute \verb+X+ asks to infer the probability $P(X|\BFe)$ where $\BFe$ is
evidence over attributes in the system. This is done using the \verb+?+
operator. The keyword \verb+unobserve+ can be used to remove evidence over an
attribute.

\begin{footnotesize}
\begin{verbatim}
request myQuery {
  // adding evidence
  mySystem.anObject.aVariable = true;
  mySystem.anotherObject.aVariable = 3;
  mySystem.anotherObject.anotherVariable = false;
  // asking to infer some probability value given evidence
  ? mySystem.anObject.anotherVariable;
  // remove evidence over an attribute
  unobserve mySystem.anObject.aVariable;
}
\end{verbatim}
\end{footnotesize}

\section{Functions}
\label{O3PRMFunc}


Functions in O3PRM are considered as tools to define attributes \cpts{}. They
replace the \cpt{} declaration by a specific syntax depending to which family
the function belongs to. There exit three kinds of functions in O3PRM. The
first kind contains built-in functions called aggregators. These functions are
used to quantify information hold in multiple reference slots. The second sort
contains deterministic functions and the third probabilistic functions. The
last two sorts of functions are not built-in functions and are implementation
specific. We only provide a generic syntax to keep uniformity between
different O3PRM implementations. All functions share the same syntax:

\begin{footnotesize}
\begin{verbatim}
<function> ::= ( "=" | "~" ) <word> "(" [ <args> ] ")"
<args ::= <word> ( "," <word> )*
\end{verbatim}
\end{footnotesize}

The use of \verb+=+ is reserved for deterministic functions and \verb+~+ for
probabilistic functions. There are only four built-in functions in the O3PRM
language that are deterministic functions called aggregators. There are five
built-in aggregators in the O3PRM language: \verb+min+, \verb+max+,
\verb+exists+, \verb+forall+ and \verb+count+.

The \verb+min+ and \verb+max+ functions require a single parameter: a list of
slot chains pointing to attributes. The attributes must all be of the same
type or share some common supertype. If the common type is not a int, then the
type's values order is used to compute the min and max values.

\begin{footnotesize}
\begin{verbatim}
class A {
  // Some declarations
  int(0,10) myMax = max([chain_1, chain_2, ...]);
  // Some declarations
  int(0,10) myMin = max(chain_1);
}
\end{verbatim}
\end{footnotesize}

If there is only one element in the list of slot chains the \verb+[]+ are
optional. The \verb+exists+ and \verb+forall+ require two parameters: a list
of slot chains and a value. As for \verb+min+ and \verb+max+, all attributes
referenced in the slot chains list must share a common type or supertype. The
value must be a valid value of that common supertype. \verb+exists+ and
\verb+forall+ attribute type must always be a boolean.

\begin{footnotesize}
\begin{verbatim}
class A {
  // Some declarations
  boolean myExists = exists([chain_1, chain_2, ...], a_value);
}
\end{verbatim}
\end{footnotesize}

The \verb+count+ aggregator behavior is dependent on it's type wich must be of
the built-in int type. Then the last value of the aggretors type is interpreted
as \emph{at least N occurence of ''value''}.

\begin{footnotesize}
\begin{verbatim}

type int(0, 5) myRange;

class A {
  myRange myCount = exists([chain_1, chain_2, ...], a_value);
}
\end{verbatim}
\end{footnotesize}

\newpage
\section{O3PRM BNF}

\begin{footnotesize}
\begin{verbatim}
<O3PRM> ::= [<header>] <compilation_unit> [(<compilation_unit>)]

<header> ::= <import>
<import> ::= import <path> ";"

<compilation_unit> ::= <type_unit> |
                       <class_unit> |
                       <system_unit> |
                       <query_unit>


<type_unit> ::= <built_in> | <user_defined>
<user_defined> ::= <basic_type> | <subtype>
<basic_type> ::= type <word> <word> "," ( "," <word> )+ ";"
<subtype> ::= type <word> extends <word> <type_spec>
<type_spec> ::= <word> ":" <word> (<word> ":" <word>)+
<built_in> ::=
  boolean <word> ";" |
  int "(" <digit>* "," <digit>* ")" <word> ";" |
  real "(" <digit>* ( "," <digit>* )+ ")" <word> ";"

<class_unit> ::= <class> | <interface>
<class> ::= class <word> [ extends <word> ] "{" <class_elt>* "}"
<class_elt> ::= <reference_slot> | <attribute> | <parameter>

<interface> ::= interface <word> [ extends <word> ]
                "{" <interface_elt> "}"
<interface_elt> ::= <reference_slot> | <abstract_attr>

<reference_slot> ::= [internal] <word> [ "[" "]" ] <word> ";"

<attribute> ::= <word> <word> [ dependson <parents> ]
                ( <CPT> | <function> ) ";"
<parents> ::= <path> ( "," <path> )*
<abstract_attr> ::= <word> <word> ";"

<CPT> ::= "{" ( <raw_CPT> | <rule_CPT> ) "}"
<raw_CPT> ::= "[" <float>+ ( "," <float>+ )+ "]"
<rule_CPT> ::= ( <word> ("," <word>)* ":" <float> ";" )+

<parameter> ::= <int_parameter> | <real_parameter>
<int_parameter> ::= "int" <word> default <integer> ";"
<real_parameter> ::= "real" <word> default <float> ";"

<function> ::= ( "=" | " ̃" ) <word> "(" [ <args> ] ")"
<args ::= <word> ( "," <word> )*

<system> ::= system <word> "{" <system_elt>* "}"
<system_elt> ::= <instance> | <affectation>

<instance> ::= <word> [ "[" digit* "]" ] <word> ";"
<instance> ::= <word> <word> "(" <parameters> ")" ";"
<parameters> ::= instanceParameter ("," instanceParameter)*
instanceParameter> ::= <word>"="(<integer>|<float>)

<affectation> ::= <path> += <word> ";" |
                  <path> = <word> ";"

<query_unit> ::= request <word> "{" <query_elt>* "}"
<query_elt> ::= <observation> | <query>
<observation> ::= ( <path> = <word> ) |
                  ( unobserved <path> )
                  ";"
<query> ::= "?" <path> ";"

<word> ::= <letter> (<letter> | <digit>)
<letter> ::= 'A'..'Z' + 'a'..'z'+ '_'
<integer> ::= <digit> <digit>*
<float> ::= <integer> "." <integer>
<digit>  ::= '0'..'9'
<path> ::= <word> [ ( "." <word> ) ]

\end{verbatim}
\end{footnotesize}

% \section{Class inheritance and interfaces}
%
% Class inheritance and interface implementation are two kinds of class
% specialization available in the O3PRM language.
%
% \subsection*{Class inheritance}
%
% The first form of specialization is called class inheritance. It consists in
% defining a class, called the subclass, by specializing another class, called
% the superclass. The syntax to declare a subclass is as follows:
%
% \begin{footnotesize}
% \begin{verbatim}
% class MySubClass extends MySuperClass {
%   // reference slots and attributs declaration
% }
% \end{verbatim}
% \end{footnotesize}
%
% A subclass inherits all the reference slots and attributes defined by its
% superclass (and those inherited by the superclass' superclass). Thus, if we do
% not want to change properties inherited, we do not need to declare them.
% However, all inherited elements are considered as encapsulated by the
% subclass, thus they can be used to define dependencies in the subclass
% attributes. The following example illustrates this principle:
%
% \begin{footnotesize}
% \begin{verbatim}
% class C {
%   boolean z {[0.3, 0.7]};
% }
%
% // A superclass
% class A {
%   C myRef;
%   boolean b {[0.2, 0.8]};
% }
%
% // A subclass of A
% class B extends A {
%   t\_state state dependson b, maRef.z
%   { [ /* ... */ ] };
% }
% \end{verbatim}
% \end{footnotesize}
%
% Another important feature of class inheritance is reference slot overloading
% and attribute overloading. It is possible to:
% %
% \begin{itemize}
% %
% \item specialize attributes or references slots;
% %
% \item changes dependencies of an inherited attribute;
% %
% \item changes the \cpt{} of an inherited attribute.
% %
% \end{itemize}
% %
% Specializing attribute consist in changing that attribute type by a subtype.
% Specializing a reference slot consists in changing the reference slot range
% type by a subclass. Finally, either changing an attribute dependencies or its
% \cpt{} is done by redeclaring the attribute.
%
% \subsection*{Interfaces}
%
% An interface is an abstract class, \ie{} it does not have any probabilistic
% dependencies. Interfaces are used to define multiple class inheritance: when a
% class implements an interface it must declare all the reference slots and
% attributes contained in the implemented interface. Eventually, the class can
% specialize them. Declaring an interface is similar to declaring a class, but
% uses the keyword \verb+interface+ and does only defined parameters without any
% default value. The following is an example of interface declaration:
%
% \begin{footnotesize}
% \begin{verbatim}
% interface I {
%   boolean b;
% }
%
% interface J {
%   boolean c;
% }
%
% class A implements I, J {
%   bool b {[0.1, 0.9]};
%   bool c dependson c {[ /* ... */ ]};
% }
% \end{verbatim}
% \end{footnotesize}
%
% Interfaces can be used by references as their range. In such case, when the
% reference slots encapsulated class is instantiated, only instance of classes
% implementing the reference slot range can be affected to it. Finally, it is
% possible to defined subinterfaces using the keyword \verb+extends+:
%
% \begin{footnotesize}
% \begin{verbatim}
% interface I {
%   bool b;
% }
%
% interface J extends I {
%   bool c;
% }
% \end{verbatim}
% \end{footnotesize}
%
% Interface inheritance only allows to specialize reference slot or attributes,
% since there is no probabilistic information in interfaces.

\nocite{WT2012}
\nocite{TGW2011}
\nocite{TW2010}
\nocite{TWG2010}

\bibliographystyle{apalike}
\bibliography{o3prm-reference}

\end{document}
